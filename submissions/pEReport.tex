
\documentclass{article}
\usepackage[utf8]{inputenc}
\usepackage{graphicx}
\usepackage{amsmath}
\usepackage{minted}
\usepackage{color}
\usepackage[hmarginratio=1:1,top=33mm,columnsep=20pt]{geometry}

\title{Performance Engineering}
\author{Bruno Cuevas}
\date{21st Nov 2016}


\begin{document}


	\maketitle
\section {Introduction}

	In this report, the final goal is to enhance the performance of a
	C program. This program uses the Laplace approach to solve a problem
	through the use of a 5-points gradient.

\section {Modifying loops}
	
	One of the keys aspects that must be taken into account is
	that matrixes in C are stored in a memory which is linear (so
	data is actually organised by rows that are placed one
	after each other). Since processors must move through the memory
	directions to find the data instance they are going to process,
	the process can get speed up only by changing the way used to
	access the memory values of the matrix.

	In the original code, the matrix was accessed firstly by columns:

	\begin{minted}[breaklines]{c}
	float A[n][m];
	float Anew[n][m];
	float y[n];
	...
	for( i=1; i < m-1; i++ ) for( j=1; j < n-1; j++){
	  Anew[j][i]=(A[j][i+1]+A[j][i-1]+A[j-1][i]+A[j+1][i])/4;
	  error=fmaxf(error,sqrtf(fabsf( Anew[j][i]-A[j][i])));
	
	}
	\end{minted}

	A change in the loop order to access firstly by rows was performed:

	\begin{minted}[breaklines]{c}
	for( j=1; j < n-1; j++) for( i=1; i < m-1; i++ ){
	  Anew[j][i]=(A[j][i+1]+A[j][i-1]+A[j-1][i]+A[j+1][i])/4;
	  error=fmaxf(error,sqrtf(fabsf( Anew[j][i]-A[j][i])));	
        }
	\end{minted}
	
	Although this could seem trivial, the speed up was actually pretty
	high, as it will be observed in the perf measures section.

		
\section {Calculation changes}

	Computers perform operations in a different way than us. For instance,
	they do not perform division as we do but employing a method
	that only requires the fundamental operations that computers perform.
	
	A possible way to speed up the process is using multiplication
	by 0.25 instead of division by 4 within the Anew[j][i] calculation.

	
	\begin{minted}[breaklines]{c}
        for( j=1; j < n-1; j++) for( i=1; i < m-1; i++ ){
          Anew[j][i] = ( A[j][i+1]+A[j][i-1]+A[j-1][i]+A[j+1][i])*0.25;
          error=fmaxf(error,sqrt(fabsf( Anew[j][i]-A[j][i])));
        }
	\end{minted}
	
	Moreover, it can be observed that $\sqrt{\|An_{j,i} - A_{j,i}\|}$
	is being calculated within the loop. Although we need that
	the error is root square of the substraction values, it is
	known that the largest error will also have the highest square root,
	so it is not necesary to calculate the square root until 
	flow gets out of the loop.

	\begin{minted}[breaklines]{c}
        for( j=1; j < n-1; j++) for( i=1; i < m-1; i++ ){
          Anew[j][i] = ( A[j][i+1]+A[j][i-1]+A[j-1][i]+A[j+1][i])/4;
          error=fmaxf(error,fabsf( Anew[j][i]-A[j][i]));
	...
	error = sqrt(error);
        }
	\end{minted}
	
\section {Double Buffer}

	Until now, the process has consisted in calculating
	the values of $An_{i,j}$ from the values of $A_{i,j}$, and
	then update the $A$ matrix as $A_{i,j}=An_{i,j}$. This process
	is time consuming since it needs to iterate all over the
	values of $An$ matrix to update $A$.

	A double buffer can avoid this loop without need to
	consume more resources. Instead of calculating the values
	of $An$ and then copy those values to $A$, the program
	will calculate the values for $A$ or $An$ depending
	on the iteration number, using the values of the other
	matrix.


	\begin{minted}[breaklines]{c}
	if(iter % 2 == 0){
	 for( j=1; j < n-1; j++) for( i=1; i < m-1; i++ ){
	   Anew[j][i] = ( A[j][i+1]+A[j][i-1]+A[j-1][i]+A[j+1][i])*0.25f;
	   error = fmaxf( error,  fabsf( Anew[j][i]-A[j][i] )  );
	 }
	} else {
         for( j=1; j < n-1; j++) for( i=1; i < m-1; i++ ){
	   A[j][i] = ( Anew[j][i+1]+Anew[j][i-1]+Anew[j-1][i]+Anew[j+1][i])*0.25f;
	   error = fmaxf( error,  fabsf( Anew[j][i]-A[j][i] )  );
         }
	}
	\end{minted}

	As we can see, when the iter value (which is the
	iteration value) is odd, values will be calculated for $An$ from $A$,
	and when iter is even, values will be calculated for $A$ from $An$.
	
\section {Parallelization through OpenMP}

	By now, the program has been using only one thread of execution. The
	use of OpenMP libraries can help to enhance the performance through
	calcuating different $An$ values at the same time. A parallel
	section was introduced.

	\begin{minted}[breaklines]{c}
	#pragma omp parallel default(none)\
	 shared(Anew,A) private(i,j) \
	reduction(max:error){
	 #pragma omp for nowait
	  for( j=1; j < n-1; j++) for( i=1; i < m-1; i++ ){
	    Anew[j][i] = ( A[j][i+1]+A[j][i-1]+A[j-1][i]+A[j+1][i])*0.25f;
	    error = fmaxf( error,  fabsf( Anew[j][i]-A[j][i] )  );
	  }
	 #pragma omp barrier
	 #pragma omp for nowait
	  for( j=1; j < n-1; j++) for( i=1; i < m-1; i++ )
	    A[j][i] = Anew[j][i];
	}

	\end{minted}

	There are two aspects that worth attention in this implementation:

	\begin{itemize}
		\item There is a reduction statement. It allows us
		to keep the error value updated to the maximum
		error within the calculation without the need
		of storing this value inside a vector that
		should be iterated to find maxima.
		\item The use of double buffer is depreciated for
		parallelization.
	\end{itemize}

\section{Compilation}

	The program can be compiled in different ways: using the default
	gcc compiler that is included in all normal Linux distros,
	or the icc compiler, property of Intel Corporation.

	Booth versions will be tested to compare if
	they produce any enhance of the performance.

\section{Statistics}

\end{document}


