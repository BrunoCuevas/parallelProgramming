\documentclass{article}
\usepackage[utf8]{inputenc}
\usepackage{graphicx}
\usepackage[utf8]{inputenc}
\usepackage{minted}
\usepackage{color}
\usepackage{graphicx}

\title{Parallel Programming. OpenMP}
\author{Bruno Cuevas, Albert Cuevas, Iñigo Morcillo}
\date{19th October 2016}

\begin{document}

\maketitle

\section {Introduction}
	In the previous task proposed within this subject, a C binary was compiled
	which was able to:
	\begin{itemize}
		\item Obtain an answer to the proposed problem (vibration of a string
		 using finite differences).
		\item Use dynamical memory allocation.
		\item Improve the performance of the binary through reducing the amount
		of memory necessary to arrive to the final answer.
	\end{itemize}
	In this task, it has been proposed to make a parallel program through the use
	of shared memory paradigm, implemented for C in OpenMP.

\section {Code}

	The program has been built using parallel sections in the iteration over
	the elements of the array for each time step. Since the program does
	only store in memory the lines that are being employed for the current
	calculation, then it is necessary to realocate the elements, task that is
	also performed in parallel.

	Since parallel operations are in 'no wait', then a barrier had to be
	introduced between the two parallel loops of the section in order to avoid
	the realocation of elements that have not been processed yet, which would
	lead to undeterminism (we could obtain different results in
	each run) and error.



\begin{minted}[breaklines]{c}
	#include <stdio.h>
	#include <stdlib.h>
	#include <math.h>
	#include <time.h>
	#include "omp.h"
	int main(int argc, char *argv[]) {
		int X = atoi(argv[1]);
		int T = atoi(argv[2]);
		// oS : operation-matrix Size
		int oS = 3;
		double *U = malloc(sizeof(double)*(X+1)*3);
		int t, x;
		double L= 0.345678, S;
		// initialize positions of matrix U
		for (x=1; x<X; x++) {
			U[x] = sin(x*M_PI/(double)X);
			U[x + X + 1] = U[x] * cos(M_PI /(double)T);
		}
		for (t=0; t<oS; t++) {
			U[t*(X+1)] = U[t*(X+1) + X] = 0.0;
		}
		// Program Body
			// tS : timeStep
			// iX : iterX
			// fstTerm : First term of the equation
			// sndTerm : Second term of the equation
			// trdTerm : Third term of the equation
			// fthTerm : Fourth term of the equation
			// iRXL : index X for relocation
			// iRTL : index T for relocation
		int tS, iX, address, iRXL;
		double fstTerm, sndTerm, trdTerm, fthTerm;



		for (tS = 1; tS < T + 2; tS ++) {
			#pragma omp parallel default(none) shared(tS, T, U, L, X) private(address, fstTerm, sndTerm, trdTerm, fthTerm, iX, iRXL)
			{
			#pragma omp for nowait
			for (iX = 1; iX < X ; iX ++) {
			address = ((X+1)) + iX;
			fstTerm = 2*(1-L)*U[address];
			sndTerm = L*U[address + 1];
			trdTerm = L*U[address - 1];
			fthTerm = -U[address - (X+1)];
			U[address + (X+1)] = fstTerm + sndTerm + trdTerm + fthTerm ;

			}
			#pragma omp barrier
			#pragma omp for nowait
			for (iRXL = 1 ; iRXL < X ; iRXL ++) {
			U[iRXL] = U[iRXL + X + 1];
			U[iRXL + X + 1] = U[iRXL + 2*(X+1)];
			}
			}
		}


		// obtain checksum of final state
			// cS : checkSum for a given value
		S = 0;
		int fP = (X+1);
		for (iX = 0; iX <= X; iX ++){
			S = S + U[fP + iX];
		}
		printf("CheckSum = %e\n", S);
		return 0;
	}

\end{minted}

	To run the code, use:

\begin{minted}[breaklines]{shell}
	$ ./e1OMP LENGTH TIME
\end{minted}
	To compile the code, use:
\begin{minted}[breaklines]{shell}
	$ gcc -Wall ejercicioOMP.c -o e1OMP -lm -fopenmp
\end{minted}


\section {Performance}

	The program has been tested in a 4 threads computer against its predecessor,
	the non-parallel binary, using the shell tool -temp-. Real execution-time
	was tested for two array sizes, 10000 and 100000.

	\includegraphics[width=\textwidth]{fig10000}
	\includegraphics[width=\textwidth]{figure100000}

	In both cases, a t-test was done to test if differences were significative.
	The p-value for the 10000 length array experiment was 0.0026, and for the
	100000 this was much lower ($10^{-15}$), so we can conclude that
	the parallel implementation has a different speed (in this case, a
	larger speed) than the not-parallel implementation.

	Since the computer where it has been tested has 4 cores, we would have
	expected a 4 times speed-up. However, it does only run 40\% faster.

	A hypothesis for this speed-up observed is that the process of opening and
	clossing threads in each time step leads to an important loss of
	speed. Another source of loss of speed in our implementation is the
	reallocation of terms in the matrix. However, it helps us saving memory,
	so we consider that it is not actually a loss of efficiency, only of speed.


\end{document}
